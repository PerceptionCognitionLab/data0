Jamie, Linda,

This is a typical debriefing for a current line of experiments.
Undergraduate and graduate research associates typically identify the
manipulation, sketch the predictions, and show what previous data
collected in the lab has indicated.  

Thanks,
Jeff

----------------------------------------------------------


Thank you for your participation.  

The experiment you performed comes from a long line of research.  In
some sense, people are quite amazing at identification.  You cab
recognize about 100,000 English words and maybe a half million
different objects.  For all of this wonderful recognition abilities,
you have a difficult time identifying a dozen different lines!  This
dramatic difference, between a half-million objects and twelve line
lengths has fascinated psychologists for fifty years.  One explanation
is that people cannot represent length, at least not in a long-term
store.  They remember seeing a line but do not store how long it is.
Another explanation is that people can remember line-length
information but there is interference from other lines.  Its like
learning names.  If there is one name to learn, its pretty easy; but
several names gets harder because they all run together.  

This is experiment is designed to test these two competing
explanations.  Some of you were in the low-interference condition.
When you learned lines, you only learned a few at a time.  This was
done to reduce interference.  Others of you were in high intereference
condition.  You learned all of the lines at once.  The intereference
theory predicts this manipulation matters but the no-storage theory
predicts that this manipulation doesn't matter.  In fact, in previous
experiments, we have found that the interference manipulation doesn't
matter; this leads to the tentative conclusion that people don't store
line-length information at all.



