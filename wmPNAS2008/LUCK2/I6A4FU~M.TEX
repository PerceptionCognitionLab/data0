Experiment: wm/luck1/
Executable: luck1

Briefing:

The experiment you will be involved in today is a memory
experiment. We want to test how good your memory is for colored
squares. In each trial of the experiment, you will see two displays of
colored squares. In one of these displays, a single square may have
changed.
 It is your task to tell when the
either a square is the same (press 'z') or different (press
'/'). These displays may have 4, 6, or 8 squares.

[draw display with squares]

In half the experiment, you will be cued on the test display; only one
of the squares
will reappear. If this square is the same color as it was in the first
display, press 'same'. If it is different,
press 'different'. This half is marked by ``Single probe''
instructions.
 
In the other half of the experiment there will be no cue. The whole
display will reappear. 
On some trials one of the squares may be different. If no squares have
changed, press 'same'. Otherwise, press different.
This half is marked by ``no probe'' instructions.

[demonstrate pie chart]

The experiment is divided into blocks. In each block, there will be a
different proportion of change and non-change trials. At the beginning
of each block, you will see a pie chart which tells you what the
proportion is. 

At the beginning of each half, there will be several practice
trials so that you can get the hang of the task. During the first set
of practice, please ask if you have any questions about the task.

Do you have any questions that I can answer now?


Debriefing:
In the memory experiment you just participated in, you were asked to
remember arrays of colored squares we are interested in estimating
memory capacity using this task. For instance, imaging you have a
fixed capacity in your working memory, which you use to remember the
squares. The chance that you'll retain the square that changed, if any
changed, is your capacity divided my the number of squares. We can use
this fact to estimate capacity. 

We are also interested in how the probability of change manipulation
affects the estimated capacity. If capacity is truly fixed, it should
not affect capacity. If the manipulation does affect our estimates of
capacity, we know that something about our model of memory is not
correct.

Any questions?
